\documentclass[12pt]{scrartcl}

\setlength{\parindent}{0pt}
\setlength{\parskip}{.25cm}

\usepackage{graphicx}

\usepackage{xcolor}

\definecolor{darkred}{rgb}{0.5,0,0}
\definecolor{darkgreen}{rgb}{0,0.5,0}
\usepackage{hyperref}
\hypersetup{
  letterpaper,
  colorlinks,
  linkcolor=red,
  citecolor=darkgreen,
  menucolor=darkred,
  urlcolor=blue,
  pdfpagemode=none,
  pdftitle={CSCE 156 Lab Handout},
  pdfauthor={Christopher M. Bourke},
  pdfsubject={},
  pdfkeywords={}
}

\definecolor{MyDarkBlue}{rgb}{0,0.08,0.45}
\definecolor{MyDarkRed}{rgb}{0.45,0.08,0}
\definecolor{MyDarkGreen}{rgb}{0.08,0.45,0.08}

\definecolor{mintedBackground}{rgb}{0.95,0.95,0.95}
\definecolor{mintedInlineBackground}{rgb}{.90,.90,1}

%\usepackage{newfloat}
\usepackage[newfloat=true]{minted}
\setminted{mathescape,
               linenos,
               autogobble,
               frame=none,
               framesep=2mm,
               framerule=0.4pt,
               %label=foo,
               xleftmargin=2em,
               xrightmargin=0em,
               startinline=true,  %PHP only, allow it to omit the PHP Tags *** with this option, variables using dollar sign in comments are treated as latex math
               numbersep=10pt, %gap between line numbers and start of line
               style=default, %syntax highlighting style, default is "default"
               			    %gallery: http://help.farbox.com/pygments.html
			    	    %list available: pygmentize -L styles
               bgcolor=mintedBackground} %prevents breaking across pages
               
\setmintedinline{bgcolor={mintedBackground}}
\setminted[text]{bgcolor={mintedBackground},linenos=false,autogobble,xleftmargin=1em}
%\setminted[php]{bgcolor=mintedBackgroundPHP} %startinline=True}
\SetupFloatingEnvironment{listing}{name=Code Sample}
\SetupFloatingEnvironment{listing}{listname=List of Code Samples}


\title{CSCE 156 -- Computer Science II}
\subtitle{Lab 9.0 - JDBC in a Webapp I}
\author{~}
\date{~}

\begin{document}

\maketitle

\section*{Prior to Lab}

\begin{enumerate}
  \item Review this laboratory handout prior to lab.
  \item Make sure that the Albums database is installed and available 
  	in your MySQL instance on CSE
  \item Review the SQL and JDBC lecture notes
  \item Review a JDBC tutorial from Oracle: \\
	\url{http://download.oracle.com/javase/tutorial/jdbc/}
\end{enumerate}

\section*{Lab Objectives \& Topics}
Following the lab, you should be able to:
\begin{itemize}
  \item Write SQL queries for use in JDBC
  \item Make a JDBC connection, query and process a result set from a database
  \item Have some exposure to a multi-tiered application and a web 
  	application server
\end{itemize}

\section*{Peer Programming Pair-Up}

To encourage collaboration and a team environment, labs will be
structured in a \emph{pair programming} setup.  At the start of
each lab, you will be randomly paired up with another student 
(conflicts such as absences will be dealt with by the lab instructor).
One of you will be designated the \emph{driver} and the other
the \emph{navigator}.  

The navigator will be responsible for reading the instructions and
telling the driver what to do next.  The driver will be in charge of the
keyboard and workstation.  Both driver and navigator are responsible
for suggesting fixes and solutions together.  Neither the navigator
nor the driver is ``in charge.''  Beyond your immediate pairing, you
are encouraged to help and interact and with other pairs in the lab.

Each week you should alternate: if you were a driver last week, 
be a navigator next, etc.  Resolve any issues (you were both drivers
last week) within your pair.  Ask the lab instructor to resolve issues
only when you cannot come to a consensus.  

Because of the peer programming setup of labs, it is absolutely 
essential that you complete any pre-lab activities and familiarize
yourself with the handouts prior to coming to lab.  Failure to do
so will negatively impact your ability to collaborate and work with 
others which may mean that you will not be able to complete the
lab.  

\section*{JDBC in a Web Application}

In this lab you will familiarize yourself with the Java Database 
Connectivity API (JDBC) by finishing a simple, nearly complete 
retrieve-and-display web application and deploying it to an 
application server (Glassfish).  The design of the webapp is 
simple: it consists of an index page that loads album data 
via Ajax (Asynchronous JavaScript and XML) and displays it in
a table.  

It is not necessary to understand the details of the application 
(the HTML, JavaScript, Servlets, or application server).  The 
main goal of this lab is to give you some familiarity with JDBC 
and exposure to a multi-tiered application and web application 
server environment.

\section*{Getting Started}

\textbf{Note:} Unless you are familiar with Java Applications Servers 
and have one installed on your laptop, you will need to use the lab 
computers for this lab.  In addition, it would be a good idea to reset 
your Albums database by rerunning the SQL script from a prior lab.

For this lab, you will need to use the JEE (Java Enterprise Edition) 
version of Eclipse, not the JSE (Java Standard Edition).  In Windows, 
click the start menu and enter ``Eclipse'', the ``Java EE Eclipse'' 
should show up, select this version.  You may use the same workspace 
as with the JSE (Java Standard Edition) version of Eclipse.

Alternatively (and recommended) you can use the Eclipse version in the 
Linux partition, which will enable you to deploy locally to your lab 
machine instead of the csce server.  To use this version of Eclipse: 

\begin{enumerate}
  \item Restart your lab computer and choose openSUSE.   Login with 
    your usual CSE credentials.
  \item Once logged into Linux, go to ``Applications'' $\rightarrow$ 
  	``Search'' and open up a ``Terminal''
  \item Launch eclipse by executing 
	
	\mintinline{text}{/usr/local/bin/eclipse &}
	
	(the ampersand launches Eclipse in the background so you can 
	still use the terminal session)
\end{enumerate}

Once you've got Eclipse running, clone the project code for this lab 
from GitHub using the URL, \url{https://github.com/cbourke/CSCE156-Lab09}.  Refer to Lab 1.0 for instructions on how to clone a 
project from GitHub.	

\section*{Activities}

\subsection*{Modifying Your Application}

\begin{enumerate}
  \item You will first need to make changes to the 
  	\mintinline{java}{unl.cse.albums.DatabaseInfo} source file.  In 
	particular, change the login and password information to your
	MySQL credentials.  You can reset these by going to \url{http://cse.unl.edu/check}.
  \item The HTML, JavaScript, etc. has been provided for you.  Feel
    free to make modifications these files, but you should know what
    you are doing as changes can break functionality in other parts
    of the application.
  \item The application will not display any album data until you have
	completed the methods in the \mintinline{java}{Album} class.
	\begin{itemize}
	  \item \mintinline{java}{public static List<Album> getAlbumSummaries()} -- 
	  This method will query the database and get a complete list of all 
	  albums in the database.  It will create and populate 
	  \mintinline{java}{Album} objects and put them in a list 
	  which will then be returned.  This method will be used to 
	  generate the album table, so it doesn't need all information, just
	  a subset (see the documentation as to what is required).  
	  You should optimize your queries to only select the relevant columns.

	  \item \mintinline{java}{public static Album getDetailedAlbum(int albumId)} --
	  this method will query the database for the specific album with 
	  the given primary key and return an \mintinline{java}{Album}
	  instance with \emph{all} relevant data (band and its members, 
	  songs, etc.) specified. 

	  \item Important: do not forget to close your database resources 
      (especially connections) after you are finished using them.  
 	  \item A \mintinline{java}{Test} class has been provided for you 
      to test your \mintinline{java}{getAlbumSummaries()} method
      which you can also adapt to test your 
      \mintinline{java}{getDetailedAlbum()} method.  You should use 
      it to debug your methods before deploying your application.
    \end{itemize}
\end{enumerate}
    
\subsection*{Deploying Your Application}

You have two options for deploying your application as a Web Archive (WAR) 
file to an application server.  

\subsubsection*{Linux: Deploying Locally}

If you chose to do this lab in the Linux version, then follow these instructions.

\begin{enumerate}
  \item Right click your project and select ``Export\ldots'', select 
	``Web'', ``WAR file'', ``Next''
  \item Click Browse and select a directory (your home directory is 
	recommended) and file to export to; name the file 
	\mintinline{text}{loginLab09.war} where \mintinline{text}{login}
	is replaced by your cse login.
  \item Return to your terminal and copy the WAR file to glassfish's 
    autodeploy directory:
    
	\mintinline{text}{cp loginLab09.war ~glassfish/domains/domain1/autodeply}
 
  \item Open a web browser (``Applications'' $\rightarrow$ ``Firefox'') 
    and go to the following url: \url{http://localhost:8080/loginLab09}
    where login is replaced with your CSE login.
\end{enumerate}

\subsubsection*{Windows: Deploying to CSCE}

If you chose to do this lab in the Windows version, then follow these 
instructions.

\begin{enumerate}
  \item Export your application in a Web Archive File (WAR): right 
    click your project and choose ``Export'' $\rightarrow$ ``Export\ldots'' 
    $\rightarrow$ ``Web'' $\rightarrow$ ``WAR file''
  \item Click ``Browse'' and select a directory to export it to, name 
    the file \mintinline{text}{loginLab09.war} where \mintinline{text}{login}
    is replaced by your cse login.  The file should now be in the directory
    you selected.
  \item Copy the WAR file to the root of your CSE directory (copy it to 
    your Z: drive) 
  \item SSH (PuTTY) into \mintinline{text}{csce.unl.edu} (\emph{not} 
    \mintinline{text}{cse.unl.edu}) using your cse login
  \item Copy the war file to the server's glassfish auto deploy 
  	directory with the command
	
	\mintinline{text}{cp loginLab09.war ~glassfish/glassfish/domains/domain1/autodeploy/}
	
  \item Make sure that the file has proper permissions by executing 
    the following from the command line:
    
    \mintinline{text}{chmod 755 ~glassfish/glassfish/domains/domain1/autodeploy/loginLab09.war}
    
    where, \mintinline{text}{login} is replaced by your cse login.
    If necessary, you may need to give glassfish a kick in the butt
    by executing:
    
    \mintinline{text}{touch ~glassfish/glassfish/domains/domain1/autodeploy/loginLab09.war}
        
  \item Open a web browser and go to the following URL:
    \url{http://csce.unl.edu:8080/loginLab09/} where 
    \mintinline{text}{login} is replaced by your CSE login.
    Unless there were problems with your code, the webapp should now work, 
    you can click on album titles or bands to be taken to another page 
    that gives further details.
\end{enumerate}

\subsection*{Completing Your Lab}

Complete the worksheet and have your lab instructor sign off on it.
Before you leave, make sure you ``undeploy'' your web app by deleting 
the WAR file from the glassfish autodeploy directory:

\mintinline{text}{rm ~glassfish/glassfish/domains/domain1/autodeploy/loginLab09.war}

\section*{Advanced Activities (Optional)}

\begin{enumerate}
  \item The album listing page utilizes a web framework called Bootstrap
  	(see \url{http://getbootstrap.com/}).  However, the album detail and 
	band detail pages do not.  Take this opportunity to learn about 
	Bootstrap and use it to add stylistic elements to these pages.
  \item Many JavaScript plugins are available to add additional 
    functionality to a plain HTML table (the ability to sort, pagination,
    column rearrangement, searching, filtering, etc.).  One of the best 
    plugins is datatables, a jQuery plugin available at 
    \url{http://datatables.net/}.  Download and incorporate datatable's 
    code into your project and add the appropriate JavaScript code 
    to make your Album table more dynamic.
\end{enumerate}

\end{document}
